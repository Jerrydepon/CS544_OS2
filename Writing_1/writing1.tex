\documentclass[10pt,draftclsnofoot,journal,compsoc,onecolumn]{IEEEtran}

\usepackage[margin=0.75in]{geometry}
\usepackage{graphicx}
\usepackage{enumerate}
\usepackage{amssymb}
\usepackage{listings}
\usepackage{titling}
\usepackage{xcolor}
\usepackage{indentfirst}

% Style of Code
\definecolor{mGreen}{rgb}{0,0.6,0}
\definecolor{mGray}{rgb}{0.5,0.5,0.5}
\definecolor{mPurple}{rgb}{0.58,0,0.82}
\definecolor{backgroundColour}{rgb}{0.95,0.95,0.92}

\lstdefinestyle{CStyle}{  
    commentstyle=\color{mGreen},
    keywordstyle=\color{magenta},
    numberstyle=\tiny\color{mGray},
    stringstyle=\color{mPurple},
    basicstyle=\footnotesize,
    breakatwhitespace=false,         
    breaklines=true,                 
    captionpos=b,                    
    keepspaces=true,                 
    numbers=left,                    
    numbersep=5pt,                  
    showspaces=false,                
    showstringspaces=false,
    showtabs=false,                  
    tabsize=2,
    language=C
}

% Title
\title{Writing 1: \\
Comparison of Processes, Threads, and CPU Scheduling among Linux, Windows, and FreeBSD}
\author{
  Chih-Hsiang Wang
}
\date{April 17th, 2018}

\begin{document}
\begin{titlepage} 
\maketitle
\begin{center}
CS544\\
Operating Systems II\\
(Spring 2018)
\vspace{50 mm}
\end{center}

% Abstract
\begin{abstract}
Comparing basic concept of processes, threads, and CPU scheduling among Linux, Windows, and FreeBSD. Then further illustrating the reasons why those similarities or differences exist.
\end{abstract}
\end{titlepage}

% Begin text
\section{Introduction}
Linux is famous for being a free software operating system, FreeBSD is also free and designed as a complete end to end system. As for windows, it is the most popular personal computer operating system up to now. The three operating system have some traits in common and some different designs for specific function. In this paper, they will be compared based on their processes, threads, and schedulers. The main purpose is not to cover detailed materials but find the important differences and discuss the reasons for the design.

\section{Comparison}
The CPU scheduling is related to the decision about which process runs, when, and for how long. The scheduler divides the finite resource of processor time between the runnable processes on a system [1]. Running time for processes includes I/O time and CPU time. In order to efficiently finish the job, the scheduler makes a sequence of instructions based on the scheduling policy (or scheduling algorithm). The objective of using proper scheduling policy is simple that it can be categorized into parts below [2]:

\begin{itemize}
  \item Maximize CPU utilization and keep CPU as busy as possible.
  \item Fair allocation of CPU.
  \item Maximize throughput. (Number of processes that complete their execution per time unit)
  \item Minimize turnaround time. (Time taken by a process to finish execution)
  \item Minimize waiting time. (Time a process waits in ready queue)
  \item Minimize response time. (Time when a process produces first response)
\end{itemize}

However, each operating system may have its own scheduling policy to deal with multiple processes and threads.

\subsection{Processes and Threads}
Processes is the fundamental unit of resource allocation. Linux, Windows, FreeBSD operating systems can communicate with other processes by a shared object or through a pipe. Besides, they can create threads and child processes [3]. By default, processes are in separate address spaces without sharing memory. There are code and data for the running programs in processes. Program counter and a set of processor registers are also common features in the three operating systems [1].

While process is the basic scheduling unit in Linux, thread is the basic scheduling unit in Windows. The working principles of process between Linux and FreeBSD are similar. However, there is a major difference in process that parent and child are inherent in Linux and FreeBSD while no inherent relationship between parent and child in Windows [4]. In Linux and FreeBSD, fork() and exec() are used to create a new process from the parent. They have tree-like structure that the first process is init with PID 1. If the parent process is terminated, the associated child processes will be forced to terminate [5]. \\

\begin{lstlisting}[style=CStyle]
int main(int argc, char **argv) 
{ char *argvNew[5]; 
 int pid;
 if ((pid = fork()) < 0) { 
 printf("Fork error\n");
 exit(1);
 } else if (pid == 0) { /* child process */ 
 argvNew[0] = "/bin/ls";
 argvNew[1] = "-l";
 argvNew[2] = NULL;
 if (execve(argvNew[0], argvNew, environ) < 0) {
  printf("Execve error\n");
  exit(1);
 }
 } else { /* parent */
 wait(pid); /* wait for the child to finish */
 }
}
\end{lstlisting}
\textbf{Figure 2.1} Use of fork() and exec() in Linux. \\

In the contrast, Windows uses CreateProcess() with 10 parameters that child process will keep running even if the parent process is closed [6]. \\

\begin{lstlisting}[style=CStyle]
STARTUPINFO info={sizeof(info)};
PROCESS_INFORMATION processInfo;
if (CreateProcess(path, cmd, NULL, NULL, TRUE, 0, NULL, NULL, &info, &processInfo))
{
    WaitForSingleObject(processInfo.hProcess, INFINITE);
    CloseHandle(processInfo.hProcess);
    CloseHandle(processInfo.hThread);
}
\end{lstlisting}
\textbf{Figure 2.2} Use of CreateProcess() in Windows. \\

The difference causes more time to create new process in Windows that CreateProcess() needs to go through API to access kernel while Linux and FreeBSD can call kernel directly [7].

About the types of process, real-time process have strict scheduling requirements to execute data in a short time period. The processing needs a continuous stream of input data to produce a continuous output [8]. Except for FreeBSD, Linux and Windows have the power to utilize real-time process to deal with the instantly changing data and almost instantaneously produce the output.

In Linux, threads are almost the same as processes except that they share the same address space. They are more light-weighted because of the sharing property. The call of exec() function in one thread will cause other threads with the same address to terminate. Unlike Linux, threads in Windows and FreeBSD are taken as separate entities from a process [1][4][7].

\subsection{Schedulers}
The data structure of each operating system is different. In Linux Kernel, the name of the scheduler are O(1) and CFS. The scheduler uses link list to accomplish the work of scheduling. Besides, CFS uses a red-black tree with O(log N) complexity. FreeBSD’s default scheduler is called ULE scheduler while Windows does not have a specific name for scheduler. Both FreeBSD and Windows use multilevel priority queue as scheduler data structure [9].

Linux kernel is non-preemptive but processes are preemptive. With the time-sharing technique, multiple processes are allowed to run concurrently. In addition, Linux kernel tends to use interactive processes when applying scheduling algorithms [1] but NT- based versions of Windows uses a CPU scheduler based on a multilevel feedback queue with 32 priority levels [7]. About the responsibility, Linux and FreeBSD use scheduler process but Windows uses kernel dispatcher which runs in the context of user process. Scheduler will choose the processes from the ready queue and dispatcher will save and load the new process into CPU [10].  

All processes can be blocked by I/O, but the methods for three systems to pick up a process are different. The process priority is important to decide which job should be finished first while working on multiple tasks. In Linux, guidelines which are called nice value are set for CPU to follow. The nice value ranges from -20 to 19 and lower nice value means giving more priority to the task [1]. FreeBSD also picks the next process with the lowest priority value. In the contrary, higher number in priority list provide higher priority for Windows [9]. Process priority of Windows ranges from 1 to 31, while Process priority of FreeBSD ranges from 0 to 127. Priority change is possible in Windows and FreeBSD by system call or by kernel. This enables them to allocate CPU resources more appropriately. Although Linux is not allowed to change priory by the methods above, the priority can be set on new processes or existing processes through command line as below [11]. \\

\begin{lstlisting}[style=CStyle]
nice -n 10 apt-get upgrade
\end{lstlisting}
\textbf{Figure 2.3} Change priority on new process in Linux. \\

\begin{lstlisting}[style=CStyle]
renice 10 -p 21827
\end{lstlisting}
\textbf{Figure 2.4} Change priority on existing process in Linux. \\

It is accepted for Linux and FreeBSD to have multiple runnable processes with the same priority. By doing so, the processor will give equal time for the processes. Compared to it, Windows will run the process one by one for a quantum each [5].

\section{Conclusion}
Comparing the processes, threads, and scheduling of the three operating system. No single operating system can prevail the others. Tradeoffs exist so people choose the appropriate operating system based on their needs. Some problems of each operating system are mentioned below. In Linux, simple scheduling policy results in a complex bug-prone implementation. It violates a basic work-conserving invariant that runnable threads may be stuck in queues while there are idle cores in the system [12]. In Windows, priority inversion that a high priority task is indirectly preempted by a lower priority task results in the violation of the priority model [13]. In FreeBSD, thrashing causes the performance of the computer to decrease because of the exclusion of most application-level processing [14].

This paper only consist of the basic and the most important concepts related to processes, threads, and schedulers. There are still many factors to consider to fully compare three operating system.

% References
\newpage
\begin{thebibliography}{14}

\bibitem{1}
L. Robert, \textit{Linux Kernel Development}. Pearson Education, Inc. 2010.

\bibitem{2}
GeeksforGeeks. (n.d.). \textit{Operating System, Process Management, CPU Scheduling - GeeksforGeeks}. [online] Available at: https://www.geeksforgeeks.org/gate-notes-operating-system-process-scheduling/ [Accessed 13 Apr. 2018].

\bibitem{3}
R. Shankar. \textit{Process behavior in Windows and Linux}. [online] Available at: https://shankaraman.wordpress.com/tag/differences-between-windows-and-linux-process/ [Accessed 13 Apr. 2018].

\bibitem{4}
M. K. Mckusick, and G. V. Neville-neil., \textit{The Design and Implementation of The FreeBSD Operating System}. 1996.

\bibitem{5}
WPI, CS502 Operating System. \textit{Processes in Unix, Linux and Windows}. 2007. pp, 15~20.

\bibitem{6}
StackOverflow, \textit{How do I call ::CreateProcess in c++ to launch a Windows executable}. 2008. [online] Available at: https://stackoverflow.com/questions/42531/how-do-i-call-createprocess-in-c-to-launch-a-windows-executable [Accessed 14 Apr. 2018].

\bibitem{7}
M. E. Russinovich, and D. A. Solomon, and A. Ionescu., \textit{Windows Internal}. 2009.

\bibitem{8}
techopedia, \textit{Real-Time Data Processing}. [online] Available at: https://www.techopedia.com/definition/31742/real-time-data-processing [Accessed 12 Apr. 2018].

\bibitem{9}
\textit{A Comparison Study of Process Scheduling in FreeBSD, Linux and Win2k}. pp, 3~7.

\bibitem{10}
\textit{Compare cpu scheduling of linux and windows}. 2015. [online] Available at: https://www.ukessays.com/essays/information-systems/compare-cpu-scheduling-of-linux-and-windows.php [Accessed 15 Apr. 2018].

\bibitem{11}
nixtutor.com, \textit{Changing Priority on Linux Processes}. 2009. [online] Available at: https://www.nixtutor.com/linux/changing-priority-on-linux-processes/ [Accessed 14 Apr. 2018].

\bibitem{12}
Lozi et al, \textit{The Linux Scheduler: a Decade of WastedCores}. 2016. [online] Available at: https://blog.acolyer.org/2016/04/26/the-linux-scheduler-a-decade-of-wasted-cores/ [Accessed 14 Apr. 2018].

\bibitem{13}
Wikipedia, \textit{Priority inversion}. 2010. [online] Available at: https://en.wikipedia.org/wiki/Priority\_inversion [Accessed 15 Apr. 2018]. 

\bibitem{14}
Wikipedia, \textit{Thrashing}. 2010. [online] Available at: https://en.wikipedia.org/wiki/Thrashing\_(computer\_science) [Accessed 15 Apr. 2018].

\end{thebibliography}

\end{document}