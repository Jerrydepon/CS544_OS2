\documentclass[english,10pt,letterpaper,onecolumn]{IEEEtran} 
\usepackage[margin=.75in]{geometry}
\usepackage{graphicx}
\usepackage[utf8]{inputenc} 
\usepackage[noadjust]{cite}
\usepackage{babel}  
\usepackage{titling}
\usepackage{listings}
\usepackage{url}

% TITLE
\title{Project 2: I/O Elevators}
\author{
  Chao-Ting Wen \hspace{.5cm}
  \and
  Chih-Hsiang Wang \hspace{.5cm}
  \and
  .Suwadi
}
\date{May 6th, 2018}

\begin{document}
\begin{titlepage} 
\maketitle
\begin{center}
CS544\\
Operating Systems II\\
(Spring 2018)
\vspace{50 mm}
\end{center}

% ABSTRACT
\begin{abstract}
Develop the solution based off the current FIFO (noop) implementation in the block directory and write a new file titled sstf-iosched.c. Then Build a LOOK I/O scheduler.

\end{abstract}
\end{titlepage}

% Begin of the text
\clearpage
\subsection*{\bf 1. The design we plan to use to implement the LOOK algorithms.}
Based on the noop-iosched.c, we first understand the basic structure of the elevator algorithm. Then we search for the information about LOOK. We know that it's head is moving in the constant direction and it will go through I/O queue to get the requests. The second step, it will change the direction and repeat the process when there is no data to read. We used a queue to check the requests and control the direction of the header.

\subsection*{\bf 2. Questions to the assignment.}
\subsection{Main point of the assignment}
The main point of this assignment is to help us be acquainted with a I/O schedule algorithm, and cultivate the ability to read and design new algorithms. Also, it help us understand the Block IO Layer, in terms of:
\begin{itemize}
	\item Different data structures involved in implementing stage.
    \item How scheduling is done on the block layers.
	\item How the LOOK, C-LOOK algorithms work in particular.
	\item How character devices differ from block devices.
    \item The use of abstraction.
\end{itemize}

\subsection{How to approach the problem}
In the first step, we spent much of time to read the materials about I/O schedulers, which helped us know more about conceptual background, essential knowledge to begin step two, analyzing the source code of the existing schedulers. Once we felt somewhat confident with the material, we began the step three, the implementation. This step was hard that we spent a lot of time finding the correct approach.

\subsection{How to ensure the solution was correct}
We successfully change the config to SSTF and make -j4. Moreover, we compile and run the scheduler without error. Besides, we print out the result at some important places to make sure each step is working as we expect. As a result, we are quiet sure our solution is correct.

\subsection{What we learn}
We learn the basic concept of I/O scheduling. However, it was a difficult process for us. We did not know where to find the instruction about how to start. We learned that how important it is to start a project without the full instruction, which is similar to what we may face in the future position. The important skill is knowing how to ask questions and searching for the crucial information. Besides, the ability to read others' code is also important for computer science students. 

\subsection{How should the TA evaluate your work? Provide detailed steps to prove correctness.}
When running the VM, it will print out the queue and the position of the head. Based on the comparison of these number, it is easy to check that our solution is correct. 

% Version Control Log
\subsection*{\bf 3. Version Control Log}
\begin{tabular}[c]{lcccr}
Date & Arthor & Commit & Messages \\\hline
05/01/2018 & ChihHsiang Wang & 989dc90d90b19108ba98f919018dddcf5eb65efd & first \\
05/05/2018 & Jerrydepon & 989dc90d90b19108ba98f919018dddcf5eb65efd & Create sstf-iosched.c \\
05/05/2018 & ChihHsiang Wang & abe504c424fc19634c81f28f8072fdc84c870a81 & add Kconfig and sstf \\
05/06/2018 & ChihHsiang Wang & a1b857784d48d4b2df02932b98935f46af4d2b91 & final work \\


\end{tabular}

% Work Log
\subsection*{\bf 4. Work Log}
\begin{tabular}[c]{lcccr}
Date & Project Progress \\\hline
04/27/2018 & Group Meeting I  \\
04/28/2018 & Study what is elevator algorithm \\
04/30/2018 & Learn how scheduling is done on the block layer \\
05/01/2018 & Group Meeting II \\
05/01/2018 & Discuss how to use the LOOK and C-LOOK \\
05/05/2018 & Finish the sstf-iosched.c \\
05/06/2018 & Delete and Rebuild the environment \\ 
05/06/2018 & Upload final work to github \\
05/06/2018 & Writing Completed \\

\end{tabular}

\end{document}