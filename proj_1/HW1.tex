\documentclass[english,10pt,letterpaper,onecolumn]{IEEEtran} 
\usepackage[margin=.75in]{geometry}
\usepackage{graphicx}
\usepackage[utf8]{inputenc} 
\usepackage[noadjust]{cite}
\usepackage{babel}  
\usepackage{titling}
\usepackage{listings}
\usepackage{url}

% TITLE
\title{Project 1: Getting Acquainted}
\author{
  Chao-Ting Wen \hspace{.5cm}
  \and
  Chih-Hsiang Wang \hspace{.5cm}
  \and
  .Suwadi
}
\date{April 13th, 2018}

\begin{document}
\begin{titlepage} 
\maketitle
\begin{center}
CS544\\
Operating Systems II\\
(Spring 2018)
\vspace{50 mm}
\end{center}

% ABSTRACT
\begin{abstract}
In the project, we set up the virtual machine of Linux Kernel. Besides of using QEMU virtual machine monitor, this project shows the steps of log of command, the explanation of flags in QEMU, the version of control log and the work log.
\end{abstract}
\end{titlepage}

% Begin of the text
\clearpage
\subsection*{\bf 1. Log of Command}
Operating within a qemu based Yocto environment.\\\\
// open the terminal\\
// login to os2 server\\
ssh os2.engr.oregonstate.edu -l wangchih\\\\
// move to the /scratch/spring2018/ and make a group \\
cd /scratch/spring2018\\
// make a group directory\\
mkdir 6\\
// move into group directory\\
cd 6\\\\
// download the necessary files from github\\
git clone git://git.yoctoproject.org/linux-yocto-3.19\\
cd linux-yocto-3.19/\\
// change the version to 3.19.2\\
git checkout tags/v3.19.2\\
//to check the version\\
git branch\\\\
// copy necessary files into group directory\\
cd ../../../..\\
cp /scratch/files/* /scratch/spring2018/6/\\\\
// build a new instance of the kernel, and make sure it will boot in the VM\\
cp config-3.19.2-yocto-standard ./linux-yocto-3.19/.config\\
make -j4\\\\
// change to the bash shell and source the file\\
bash\\
source environment-setup-i586-poky-linux\\
// launch qemu in debug mode, then the process will halt until next cmd 
% need to change -drive file=core-image-lsb-sdk-qemux86.ext4
qemu-system-i386 -gdb tcp::5506 -S -nographic -kernel bzImage-qemux86.bin -drive file=core-image-lsb-sdk-qemux86.ext4,if=virtio -enable-kvm -net none -usb -localtime --no-reboot --append "root=/dev/vda rw console=ttyS0 debug"\\\\
// (second terminal)\\
// to run the VM, open new terminal for GDB as remote target at the port\\
// specified above, and continue the process.\\ 
cd /scratch/spring2018/6\\
// open gdb\\
gdb\\
// connect to the port\\
(gdb) target remote :5506\\
// use "continue" to run the qumu in previous terminal.\\
(gdb) continue\\

\subsection*{\bf 2. Explanation Of Flags}
qemu-system-i386 -gdb tcp::5506 -S -nographic -kernel bzImage-qemux86.bin -drive file=core-image-lsb-sdk-qemux86.ext4,if=virtio -enable-kvm -net none -usb -localtime --no-reboot --append "root=/dev/vda rw console=ttyS0 debug"\\\\
-gdb dev\\
Wait for gdb connection on device dev. Typical connections will likely be TCP-based, but also UDP, pseudo TTY, or even stdio are reasonable use case.\\\\
-S\\
Do not use CPU in the begining.\\\\
-nographic\\
The grafical output can be disabled with thia simple QEMU command line.\\\\ 
-kernel bzImage\\
It will provide the Linux kernel image with bzImage. The kernel can be either a Linux kernel or in multiboot format.\\\\ 
-drive option[,option[,option[,...]]]\\
Define a new drive. "File" option defines which disk image to use with this drive. "If" option defines on which type on interface the drive is connected.\\\\
-enable-kvm\\
Enable KVM full virtualization support. This option is only available if KVM support is enabled when compiling.\\\\ 
-net none\\
No network devices should be configured. It is used to override the default configuration (-net nic -net user) which is activated if no -net options are provided.\\\\
-usb\\
Enable the USB driver (will be the default soon)\\\\
-localtime\\
For correct date in MS-DOS or Windows.\\\\
-no-reboot\\
Exit instead of rebooting.\\\\ 
-append cmdline\\
Use cmdline as kernel command line.\\

\subsection*{\bf 3. Version Control Log}
\begin{tabular}[c]{lcccr}
Date & Arthor & Commit & Messages \\\hline
04/13/2018 & Jerrydepon & 886282d13fb4bec43a40cece15fe152d2378fe54 & first commit \\
04/13/2018 & Jerrydepon & e7a192015611e9fc6c50934467a280ef316487a0 & add IEEEtran.cls \\
04/13/2018 & Jerrydepon & 1c545c19d7c723b40ce1c1033b4d2cef3a69e84f & first upload Latex \\
04/13/2018 & Jerrydepon & 795ffe270ffb1d5baf4a744731b9ee94bcca7262 & Delete IEEEtran.cls \\
04/13/2018 & Jerrydepon & 523cf5514e322f91c089232cb6d03517727200cc & fix Latex

\end{tabular}

\subsection*{\bf 4. Work Log}
\begin{tabular}[c]{lcccr}
Date & Project Progress \\\hline
04/07/2018 & Group Meeting I  \\
04/07/2018 & Boot VM \\
04/07/2018 & Find the meaning of flags \\
04/08/2018 & Study the Semantic Rule in Overleaf \\
04/09/2018 & Start Writing LaTex with Overleaf \\
04/11/2018 & Group Meeting II \\
04/13/2018 & Upload to github \\
04/13/2018 & Writing Completed \\

\end{tabular}

% references, IEEE format
\subsection*{\bf References}
\begin{enumerate}
\item S.Weil, "QEMU Binaries for Windows and QEMU Documentation" Dec 21, 2017 {\textit {QEMU Emulator User Documentation}} [Online] Available: https://qemu.weilnetz.de/w64/2012/2012-12-04/qemu-doc.html [Accessed on Apr 07, 2018]
\end{enumerate}

\end{document}
